\documentclass[oneside]{book}
\usepackage{datetime}
\usepackage{amsmath}
\usepackage{mathtools}
\usepackage[color={[gray]{1}}]{draftwatermark}
\usepackage{xcolor}
\usepackage{amsthm}
\usepackage{amsfonts}
\usepackage{amssymb}
\usepackage[theorems,breakable]{tcolorbox}
\usepackage{hyperref}
\hypersetup{
	colorlinks=true,
	citecolor=red!80!black,
}

\begin{document}
	%Assorted Document Settings & Initial Configurations
	\pagenumbering{arabic}
	\newtcbtheorem[number within=chapter]{theo}{Theorem}{breakable,colback=blue!5,colframe=blue!40!black}{th}
	\newtcbtheorem[number within=tcb@cnt@theo]{cor}{Corollary}{breakable,colback=blue!20,colframe=blue!80!black}{cr}
	\newtcbtheorem{defn}{Definition}{theorem name,colback=orange!20,colframe=orange!80!black}{defn}
	\newtcolorbox{sk}{colback=green!5,colframe=green!40!black,title=Sketch}
	\newtcolorbox{rmk}{colback=yellow!5,colframe=yellow!40!black,title=Remark}
	\newtcolorbox{pr}{colback=orange!5,colframe=orange!80!black,title=Proof}
	\SetWatermarkText{The rightful author of this document can be found at N00bcak.github.io.}

% I need to figure out what to do with this snippet, TeXStudio made it so ugly because it messed up the positions of the percentage signs.	
	%	\begin{theo}{}{T1.1}
		%		A theorem.
		%		\begin{sk}
			%			{Lorem ipsum dolor sit amet this is a proof.}
			%		\end{sk}   
		%		\begin{pr}
			%			{Lorem ipsum dolor sit amet this is a proof.}
			%		\end{pr}   
		%	\end{theo}
	%	\begin{cor}{Title}{C1.1}
		%		This is a corollary of Theorem \ref{th:T1.1}.
		%	\end{cor}
	%	\begin{defn}{title}{D1.1}
		%		This is a definition.
		%	\end{defn}
	
	%Title
	\title{Linear Algebra Organised}
	\author{N00bcak}
	
	\newdate{date}{21}{03}{2022}
	%\date{\displaydate{date}}
	\maketitle
	
	\chapter*{Preface}
	
	I hope that this set of notes will provide a comprehensive look at some of the most fundamental aspects of linear algebra. At the very least, because I wrote these notes principally FOR understanding, they should be a much easier read than Hoffman and Kunze etc.
	
	These notes assume a familiarity with high school mathematics and a little bit of group theory, because those concepts can easily be googled or thought through within a few minutes. Aside from that, for theorems and lemmas in general I will offer very brief exposition before explaining the broad structure of the proof (colored in green), before finally painstakingly providing a detailed proof (provided in orange).
	
	I don't really have much else to say so let's just get into it.
	
	{
		\hypersetup{linkcolor=black}
		\tableofcontents
	}
	\chapter{The Basics}
	
	This chapter is responsible for laying the foundations of whatever you are going to read in these notes, that is, the fundamental language with which linear algebra is expressed. 
	
	I will be using a \href{https://simple.wikipedia.org/wiki/Field_(mathematics)}{field} in most of my definitions because they cover the full range of real and complex arithmetic.
	
	\section{Systems of Linear Equations}
	Among the core objects of study in linear algebra are linear equations and systems thereof, which frequently appear in science and mathematics.
	
	While there is obviously far more to studying linear algebra, of particular interest in this section are the solution(s) to these systems which we will explore in detail.
	
	We first get the pesky definitions out of the way:
	\begin{defn}{Linear Equation}{Dlineq}
		Where F is any field, $(c_1,c_2,\ldots,c_{n+1})$ are constants in F while $(x_1,x_2,\ldots,x_n)$ are variables in F, a \textbf{linear equation} in \textbf{n variables} is any equation of the form
		\begin{equation}
			\label{eq:lineq}
			c_1x_1+c_2x_2+\ldots+c_nx_n=c_{n+1}
		\end{equation}
	\end{defn}
	\textit{Remark.} Often you will notice that the left hand side of \ref{eq:lineq} is called a \textbf{linear combination} of the variables $(x_1,x_2,\ldots,x_n)$.
	
	\begin{defn}{System of Linear Equations}{Dsyslineq}
		A \textbf{system of m linear equations in n unknowns} is a collection of m linear equations involving n variables. (An example is provided below):
	\begin{equation}
		\label{eq:syslineq}
		\left\{
		\begin{alignedat}{4} 
			c_{1,1}x_1 & +{} &  c_{1,2}x_2 & +{} & \ldots & +{} & c_{1,n}x_n & = c_{1,n+1} \\
			c_{2,1}x_1 & +{} &  c_{2,2}x_2 & +{} & \ldots & +{} & c_{2,n}x_n & = c_{2,n+1} \\
			\vdots \\
			c_{m,1}x_1 & +{} &  c_{m,2}x_2 & +{} & \ldots & +{} & c_{m,n}x_n & = c_{m,n+1} \\
		\end{alignedat}
		\right.
	\end{equation}
	\end{defn}
	\textit{Remark.} If $c_{1,n+1}=c_{2,n+1}=\ldots=c_{m,n+1}=0$, our system of linear equations is \textbf{homogeneous}.

	\begin{defn}{Solutions of a System}{Dsollineq}
		For any system of linear equations (\ref{eq:syslineq}), if a collection of n \textbf{values} (also known as an n-tuple) $D=(d_1,d_2,\ldots,d_n)$ all in F are such that the system is satisfied by setting $x_1=d_1, x_2=d_2, \ldots, x_n=d_n$ is called a \textbf{solution of the system of linear equations}.
		
		The collection of all such n-tuples that satisfy the system of equations is also known as a \textbf{solution set} of the system. We will mathematically describe solution sets in a later section.
	\end{defn}
	
	By playing around with the definitions above (perhaps by substituting the constant values for numbers), a few facts about the solutions of systems should become apparent. Unfortunately these same observations were made by Gauss centuries earlier so you will not become a famous mathematician for your endeavours :(.
	
	\begin{theo}{Gaussian Elimination}{T1.1}
		If one system of linear equations is manipulated into another by the following actions:
		\begin{enumerate}
			\item Rescaling (Multiplication of both sides of one equation by a non-zero constant.)
			\item Row Combination (Addition of a non-zero multiple of equation i to both sides of another equation j)
			\item Swapping (Swapping of two equations)
		\end{enumerate}
		The solution set of the new system of linear equations is the same as the original system, that is, these two systems of linear equations are \textbf{equivalent}.
	\end{theo}
	\begin{pr}
		Let a system of n linear equations be such that it has a solution $D=(d_1,d_2,\ldots,d_n) \in S$ where $d_1,d_2,\ldots,d_n$ are all in F and $S$ is the solution set of the original system of linear equations. Similarly, let $S'$ be the solution set of the new system of linear equations.
	\begin{enumerate}
		\item If $D$ is a solution of an equation i then $D$ satisfies
		
		\begin{equation*}
			\sum_{k=1}^{n}{c_{i,k}x_k}=c_{i,n+1}
		\end{equation*}
		
		Hence if we multiplied both sides of this equation by a constant $g$ in F, obviously
		
		\begin{equation*}
			g(\sum_{k=1}^{n}{c_{i,k}x_k})=g(c_{i,n+1})
		\end{equation*}
		is satisfied by $D$. This implies that $S \subseteq S'$.
		
		By multiplying the above equation by $\frac{1}{g}$ in F, we get 			\begin{equation*}
			g(\sum_{k=1}^{n}{c_{i,k}x_k})=g(c_{i,n+1})
			\implies 
			\sum_{k=1}^{n}{c_{i,k}x_k}=c_{i,n+1}
		\end{equation*}
		so we see that every solution $D' \in S'$ satisfying the new system of equations also satisfies the original system of equations. So $S' \subseteq S \implies S=S'$.
	\item If $D$ satisfies equations i and j then it should satisfy
	\begin{equation}
		\label{eq:rowcomb}
		g(\sum_{k=1}^{n}{c_{i,k}x_k})+\sum_{k=1}^{n}{c_{j,k}x_k}=g(c_{i,n+1})+c_{j,n+1}
	\end{equation}

	Conversely if $D$ satisfies \ref{eq:rowcomb} then it should satisfy	
	\begin{equation*}
		\begin{split}
			g(\sum_{k=1}^{n}{c_{i,k}x_k})+\sum_{k=1}^{n}{c_{j,k}x_k}		-g(\sum_{k=1}^{n}{c_{i,k}x_k})&=g(c_{i,n+1})+c_{j,n+1} \\
			&-g(c_{i,n+1}) \\
			\implies \sum_{k=1}^{n}{c_{j,k}x_k}=c_{j,n+1} &
		\end{split}
	\end{equation*}
	
	\item Nothing fundamentally changes about the system of equations except the order in which the equations are written. If $D$ satisfies all equations in the system, it will still do so after two equations are swapped. $\blacksquare$
	\end{enumerate}
	\end{pr}
	\textit{Remark.} Gaussian Elimination is why we can solve systems of linear equations manually via an 'eliminate-and-substitute' strategy, the solving technique taught in secondary/high schools that you should be familiar with.
	
	\textit{Remark.} The proof given above elides an explicit statement of the invertibility of each of these operations. This will be clarified in the next section.
	
	\newpage
	
	\section{Matrices}
	
	Another building block of the mathematical language of linear algebra is the matrix, which among various other functions provides a neat shorthand with which we can describe systems of linear equations.
	
	\begin{defn}{Matrix}{Dmatrix}
		Generally, where $i,j,m,n \in \mathbb{Z^+}$, an $m \times n$ \textbf{matrix} A is an array of \textit{numerical entries} with $m$ rows and $n$ columns. We denote the entry in the $i^{th}$ row and $j^{th}$ column as $A_{ij}$. 
		
		However, an alternative, more formal definition is possible:
		
		(Hoffman and Kunze) An $m \times n$ \textbf{matrix} A is a function mapping the ordered pairs $(i,j), 1 \leq i \leq m, 1 \leq j \leq n$ to numbers in a field F.
	\end{defn}
	We should also note that matrices come equipped with the following operations:
	\begin{enumerate}
		\item Addition: 
			\begin{align*}
				(A+B)_{ij}=A_{ij}+B_{ij}
			\end{align*}
		where A and B are any $m \times n$ matrices.
		\item Multiplication: 
			\begin{align*}
				(AB)_{ij}=\sum_{k=1}^{r}{A_{ik}B_{kj}} 
			\end{align*}
		Note that A must have $r$ columns while B must have $r$ rows. If A is a $q \times r$ matrix while B is a $r \times s$ matrix, then AB is a $q \times s$ matrix.
		\item Transposition:
		\begin{align*}
			(A^t)_{ij}=A_{ji}
		\end{align*} 
		where A and B are any $m \times n$ matrices.
	\end{enumerate}

	There also exist special cases of matrices important enough to have their own names (though we will only define the relevant ones now):
	
	\begin{defn}{Column/Row Vector}{Dcolrow}
		A \textbf{column vector} is an $n \times 1$ matrix.
		
		A \textbf{row vector} is a $1 \times n$ matrix.
	\end{defn}
	\textit{Remark.} This is what most people think of when they hear/say a vector. Naturally, a vector is more general than that
	\begin{defn}{Zero Matrix}{Dzeromat}
		A \textbf{zero matrix} $0^{m \times n}$ is any matrix whose entries are \textbf{all} zero.
	\end{defn}
	\begin{defn}{Square Matrix}{Dsqmat}
		A \textbf{square matrix} is an $n \times n$ matrix.
	\end{defn}
	\begin{defn}{Identity Matrix}{Dimat}
		A \textbf{identity matrix} is an $n \times n$ matrix whose entries on the main diagonal are 1 and all other entries are zero.
		
		(Using \href{https://en.wikipedia.org/wiki/Kronecker_delta}{Kronecker Delta}) An \textbf{identity matrix} $I^{n \times n}$ is such that $I_{ij}=\delta_{ij}$.
	\end{defn}

		


\end{document}